\documentclass[a4paper,11pt]{article}

\title{Master's thesis project specification}
\author{
    Student: David Nilsson\\
    D Programme, KTH\\
    \texttt{davnils@kth.se}
    \and
    Supervisor: Olov Engwall\\
    Professor, KTH\\
    \texttt{engwall@kth.se}
    \and
    Supervisor: Anders Lindgren\\
    CEO, Optistring Technologies AB\\
    \texttt{anders@optistring.com}
}

\usepackage[english]{babel}
\usepackage[T1]{fontenc}
\usepackage[utf8]{inputenc}
\usepackage{listings, babel}
\lstset{breaklines=true,basicstyle=\ttfamily}
\usepackage[margin=2cm]{geometry}
\usepackage{verbatim}

\begin{document}

\maketitle

\section*{Problem definition}
\subsection*{Background}
\subsection*{Scientific question}

\section*{Project goals}
\subsection*{Motivation}
\subsection*{Evaluation}

\section*{Previous work}

\section*{Method}

\section*{Project scope}
\subsection*{Available resources}
\subsection*{Limit of scope}

\section*{Project plan}
\subsection*{Partial goals}
\subsection*{Time plan}


% PREVIOUSLY WRITTEN
\begin{comment}
\section*{Background}
PV (photovoltaic) systems are collections of solar panels connected to power
inverters, a product which transforms the DC power from solar panels to the grid (AC power).
Optistring is a startup building digital inverters which are capable of achieving
better efficiency than traditional systems.
Part of the existing product is a stream of data containing measurements from all
solar panels in the system, making analysis of the system in various aspects possible.

The purpose of this Master's thesis project is to build a software solution which
is capable of detecting failures in PV systems.
Part of this involves formulating the different kind of failures that may occur,
developing a solution, and research of how failures should be presented.

\section*{Scientific question}

\subsubsection*{The problem}
Two different kinds of failure conditions in solar panels are relevant: instantaneous power loss due to faulty solar panels, and gradual power degradation over time.
The problem is how to detect these efficiently in the presence of different system configurations, but also geographic and thermal dependencies (all systems deliver different power curves). 

\subsubsection*{Previous work}
The company itself is a startup founded on research in power electronics, and does in some ways present a unique opportunity in performing this failure detection.
One benefit of this system is real-time data from individual solar panels with several sources of information.
It is of interest to detect failures since otherwise manual inspection is required and more fine-grained power degradation might not be easily detectable.

There has been research into failure detection and the possible outcomes, such as: \\

\noindent
\emph{Monitoring and remote failure detection of grid-connected PV systems based on satellite observations} Drews, Keizer.\\
\noindent
\emph{Experimental studies of failure detection methods in PV module strings} Takashima, Yamaguchi.

\subsubsection*{Research method}
The problem should be solved by surveying possible approaches to performing classification in continuously changing measurements.
This includes understanding how statistical methods are applicable to the problem of classifying the state of solar panels.
Additionally, the results should be validated by performing simulations in different conditions that are present in real life.

\subsubsection*{Possible outcomes}
There are several possible outcomes. Firstly, the detection might be limited to simple cases, such as when there is a transient power failure.
Secondly, the more general detection of gradual power loss might be limited to certain conditions, due to reasons such as limited statistical significance in detection.

\subsubsection*{Evaluation}
The success of the project can be evaluated by checking if the relevant failure conditions are detected during different simulations.

\section*{The Student's background}
I am currently studying the theoretical computer science track, and have an interest in fields such as statistics and software development.
Previously, I have worked at Optistring where I gained knowledge in the characteristics of solar panels, embedded software development, and data processing.

\section*{Supervisor at the company}
The main point of contact at the company is Anders Lindgren, CEO, who has a background in electronic engineering, software development, and general management.
In the case of deeper technical issues I have support from a PhD student in power electronics and several software developers.
In general I find these people very knowledgeable in the field of PV power equipment.

\section*{Project scope}
As of today, there are data feeds providing real time measurements of several PV farms in Sweden.
This data is accumulated in a central database and can be analyzed.
Hence the project can be done on top of the existing work.
\end{comment}

\end{document}
