\documentclass[a4paper,11pt]{report}
\pdfoutput=1

\usepackage[english,swedish]{babel}
\usepackage[T1]{fontenc}
\usepackage[utf8x]{inputenc}
\usepackage{listings, babel}
\usepackage{graphicx}
\usepackage[colorlinks=true,linktoc=page]{hyperref}
\usepackage[nonumberlist]{glossaries}
\usepackage{subcaption}
\lstset{breaklines=true,basicstyle=\ttfamily}
\usepackage[margin=2cm]{geometry}
\usepackage{lipsum}

\selectlanguage{english}


\newcommand{\image}[4]{
  \begin{figure}[here]
  \centering
  \includegraphics[width=10cm]{images/#1} 
  \caption[#3]{#4}
  \label{fig:#2}
  \end{figure}
}


\title{Fault detection in photovoltaic systems}
\author{David Nilsson, davnils@kth.se}

\newglossaryentry{MPP}
{
  name=MPP,
  description={maximum power point corresponding to an optimal load}
}

\newglossaryentry{MPPT}
{
  name=MPPT,
  description={maximum power point tracking}
}

\newglossaryentry{solar-module}
{
  name=solar module,
  description={An enclosed box containing several interconnected solar cells}
}

\newglossaryentry{iv-curve}
{
  name=I-V curve,
  description={Plot of panel current as a function of panel voltage}
}


\makeglossaries
\glsaddall

\begin{document}
\pagenumbering{gobble}
\maketitle
\newenvironment{abstractpage}
  {\cleardoublepage\vspace*{\fill}\thispagestyle{empty}}
  {\vfill\cleardoublepage}
\newenvironment{polyAbstract}[1]
  {\bigskip\selectlanguage{#1}%
   \begin{center}\bfseries\abstractname\end{center}}
  {\par\bigskip}

\begin{abstractpage}
\begin{polyAbstract}{english}
This master's thesis concerns three different areas in the field of fault detection in PV systems.
Previous studies have concerned homogeneous systems with a large set of parameters being observed, while this study is focused on a more restrictive case.
The first problem is to discover immediate faults occuring in solar panels.
A new online algorithm is developed based on similarity measures within a single installation.
It performs reliably and is able to detect all significant faults over a certain threshold.
The second problem concerns measuring degradation over time.
A modifed approach is taken based on repetitive conditions, and performs well given certain assumptions.
Finally the third problem is to differentiate solar panel faults from partial shading.
Here a clustering algorithm DBSCAN is applied on data in order to locate clusters of faults in the solar plane, demonstrating good performance in certain situations.

\end{polyAbstract}

\begin{polyAbstract}{swedish}
Det här är en uppsats på master-nivå inom fältet feldetektering av fotovoltaiska system.
Tidigare studier har fokuserat på homogena system med en större mängd observerade parametrar, vilket generaliseras i den här studien.
Den första delen är feldetektering av snabba felförlopp i solpaneler.
En ny algoritm presenteras baserad på grader av likhet inom en enskild solpanelsinstallation.
Den presterar väl och är kapabel att hitta alla singifikanta fel över en viss nivå.
Den andra delen består av att mäta degradering av solpaneler över tid.
En variant av tidigare resultat presenteras som baseras på upprepande förhållanden, vilket presterar väl givet vissa antaganden.
Slutligen hanteras detektering av partiell skuggning och urskiljning av detta från riktiga panelfel.
Lösning är en algoritm DBSCAN som hittar kluster av data i solplanet och den påvisar god prestanda i vissa situationer.

\end{polyAbstract}
\end{abstractpage}

\selectlanguage{english}


\tableofcontents

\chapter*{Preface}
\addcontentsline{toc}{chapter}{Preface}
TBD

\printglossaries
\cleardoublepage
\addcontentsline{toc}{chapter}{\listfigurename}
\listoffigures
\clearpage
\pagenumbering{arabic}

\chapter{Introduction}
TBD

\section{Overview of fault detection}
TBD

\section{Intended readers}
The main target audience interested in this thesis are companies building products in the field of PV power electronics.
Fault detection in solar panels is an active research area and continually evolves, demonstrating an academic interest as well.
The main party interested in the results is of course the supervising company, whom are likely to
integrate a functional solution into their final product.
While being a fairly specialized thesis, it should be comprehensible to anyone with a basic background in statistics.

\section{Scope of this thesis}
The most important scope limitation is that systems should only be studied passively, i.e. the available measurements are used to detect faulty panels.  
Some technical constraints are present as well:
\begin{itemize}
\item The solar panels have 60 or 72 cells built of mono- or poly-crystalline silicon

\item The installations have 14 to 24 panels connected within a small geographic area

\item The available measurements (from each panel) are:
$U$ \footnote{Panel voltage at an optimal load},
$I$ \footnote{Panel current at an optimal load} and
$T_{module}$ \footnote{Low resolution temperature measurement from within the PV inverter}.

\end{itemize}

This reflects the consumer market of smaller panel installations with some of the most common panel types.
The type of solar panels constrains the possible output power range and defines appropriate parameters for testing.
Due to time constraints, the simulation of solar panels will largely be based on existing data streams.
These will be used a basis of generating approximate individual panel curves, limiting the amount of effort required.

The supervising provide data feeds providing real time measurements of several PV installations in Sweden.
This data is accumulated in a central database and can be analyzed.
This data has been used in order to verify real-life performance of the classification system, primarily to verify that working systems are not classified as faulty.

Additionally there are other sources that can be used to build realistic simulation models, such as measured solar energy over several years.

\section{Ethical considerations}
%this section could possibly be integrated into the overview section
%it should deal with sustainability, and argue for why the results of this thesis have a direct impact (with proper sources)
TBD


\chapter{Background}
TBD

\section{Theory of a solar cell}
TBD

\image{trina-60cell-iv-alpha.png}{trina-iv-curve}{Trina P05 IV-curve characteristic}{
  TBD
}

\image{solar-module-generic.jpg}{solar-module-generic}{Generic solar module}{
  Back and front of a generic 54-cell Silicon solar module. Every cell and the vertical interconnections are clearly visible.
}

\section{Systems of solar panels}
TBD

\section{The role of PV inverters}
TBD

\section{Solar power in practice}
TBD


\chapter{Theory of fault detection}
TBD
%mention OCPD and other types of existing protection devices
%also bind these to the insufficient protection (as shown by the thesis and others)
%provides a great motivation: both environmental and safety improvements through this work.

\section{Issues in photovoltaic systems}
TBD

\section{Partial shading of of solar panels}
TBD

\section{Approaches to fault detection}
TBD

\section{Measuring degradation}
TBD

\bibliographystyle{abbrv}
\bibliography{../ref.bib}

\end{document}
