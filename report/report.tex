\documentclass[a4paper,11pt]{report}
\pdfoutput=1

\usepackage{algpseudocode}
\usepackage{algorithm}
\usepackage[british,english]{babel}
\usepackage[T1]{fontenc}
\usepackage[utf8x]{inputenc}
\usepackage{listings, babel}
\usepackage{graphicx}
\usepackage[colorlinks=true,linktoc=page]{hyperref}
\usepackage[nonumberlist]{glossaries}
\usepackage{subcaption}
\lstset{breaklines=true,basicstyle=\ttfamily}
\usepackage[margin=2cm]{geometry}

\selectlanguage{english}

\title{Report on parameterized complexity and subgraph counting}
\author{David Nilsson, davnils@kth.se \\ Advanced Individual Course In Computer Science (DD2464)}

\newglossaryentry{MPPT}
{
    name=MPPT,
    description={Maximum Power Point Tracking}
}

\makeglossaries
\glsaddall

\begin{document}
\pagenumbering{gobble}

\maketitle
\abstract{
  This blargh is part of the examination in the project course Advanced Individual Course In Computer Science.
  It consists of two main parts: a text on parameterized complexity theory, and a description of a recent result related to counting thin subgraphs.
  The first part is an introductory text explaining common complexity classes, known results, and lower bounds.
  In addition to describing the counting algorithm there is also an introduction to counting structural properties in graph theory.
  Finally an implementation of the counting algorithm is analyzed, including correctness verification and benchmarks.
}

\tableofcontents
\cleardoublepage
\pagenumbering{arabic}

\printglossaries

\chapter{blargh1}
blargh

\section{blargh}

\chapter{blargh2}
blargh

\section{blargh}
blargh


\begin{thebibliography}{10}

\bibitem{BHKK13}
A.~{Bj{\"o}rklund}, P.~Kaski, and {\L}.~Kowalik.
\newblock Counting thin subgraphs via packings faster than meet-in-the-middle time.
\newblock {\em arXiv} preprint arXiv:1306.4111, 2013.

\bibitem{FG06}
J.~Flum, and M.~Grohe.
\newblock Parameterized complexity theory.  Vol. 3.
\newblock Heidelberg: Springer, 2006.

\bibitem{FG04}
J.~Flum, and M.~Grohe.
\newblock The parameterized complexity of counting problems.
\newblock {\em SIAM J. Comput.}, 33(4):892--922, 2004.

\end{thebibliography}

\end{document}
